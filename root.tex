\documentclass{book}
\usepackage{graphicx}
\usepackage[a4paper, left=2cm, right=2cm, top=2.5cm, bottom=2cm]{geometry}
\usepackage[english]{babel}
\usepackage[small]{titlesec}
\usepackage{subcaption} 
\usepackage[hidelinks]{hyperref}
\usepackage{listings}
\usepackage{enumitem}
\usepackage{sectsty}
%\usepackage[light]{CormorantGaramond}
\usepackage{fancyhdr}
\usepackage{amsmath}
\usepackage{footnote}
\usepackage{tgadventor}
\usepackage{titling}
\usepackage{tcolorbox}
\usepackage{multirow}
\usepackage{booktabs}
\usepackage{makecell}
\usepackage{algpseudocode}
\usepackage{algorithm}
\usepackage{mathtools}
\usepackage{xcolor}
\usepackage{amsfonts}
\usepackage{mdframed}
\usepackage{amssymb}
\usepackage{titlesec}
\definecolor{sfondo_gray}{HTML}{f0ecec}
\usepackage{tocloft}
\usepackage{titlesec}
\usepackage{fancyhdr}
\usepackage{hyperref}
\usepackage{amsfonts} % for "\mathbb" macro
\usepackage{amsmath,amsfonts}
\usepackage{xcolor}
\usepackage{xargs}                      % Use more than one optional parameter in a new commands
\usepackage[pdftex,dvipsnames]{xcolor}  % Coloured text etc.
% 
\usepackage[colorinlistoftodos,prependcaption,textsize=tiny]{todonotes}
\newcommandx{\info}[2][1=]{\todo[linecolor=OliveGreen,backgroundcolor=OliveGreen!25,bordercolor=OliveGreen,#1]{#2}}


% Customize the Table of Contents
\renewcommand{\cftchapfont}{\normalfont\bfseries}
\renewcommand{\cftsecfont}{\normalfont}
\renewcommand{\cftsubsecfont}{\normalfont\itshape}

\renewcommand{\cftchappagefont}{\normalfont\bfseries}
\renewcommand{\cftsecpagefont}{\normalfont}
\renewcommand{\cftsubsecpagefont}{\normalfont\itshape}

\setlength{\cftbeforechapskip}{0.5em}
\setlength{\cftbeforesecskip}{0.25em}
\setlength{\cftbeforesubsecskip}{0.15em}

% Title formatting
\titleformat{\chapter}[block]{\normalfont\huge\bfseries}{\thechapter.}{1em}{}
\titleformat{\section}[block]{\normalfont\Large\bfseries}{\thesection.}{1em}{}
\titleformat{\subsection}[block]{\normalfont\large\itshape}{\thesubsection.}{1em}{}


% Elimina le pagine vuote
\usepackage{etoolbox}
\makeatletter
\patchcmd{\cleardoublepage}{\hbox{}}{}{}{}
\patchcmd{\cleardoublepage}{\thispagestyle{empty}}{}{}{}
\makeatletter

\newcommand{\chapternumberbox}[1]{%
  \colorbox{gray!50}{%
    \parbox[c][3cm][c]{3cm}{%
      \centering\Huge\sffamily #1
    }%
  }
}
\titleformat{\chapter}[display]
  {\normalfont\sffamily\LARGE}
  {\chapternumberbox{\thechapter}}
  {0pt}
  {\Huge}

\input{setup.tex}
\input{macro.tex}

\setlength{\droptitle}{-5em}   % regola la posizione del titolo
\pretitle{\begin{center}\Huge}   % imposta la dimensione della font del titolo
\posttitle{\end{center}}
%\lstset{language=C}
\renewcommand\bfdefault{bx}
\sectionfont{\fontsize{20.74}{15}\selectfont}
\subsectionfont{\fontsize{17.28}{15}\selectfont}
\subsubsectionfont{\fontsize{12}{15}\selectfont}
\title{}

\author{}
\date{\today}

\begin{document}
\raggedbottom % evita spazi verticali extra

\pagestyle{fancy}
\fancyhf{}
\rfoot{\thepage}
\lhead{\quad \leftmark}
\rhead{ \quad \rightmark}
% Fancyhdr settings
\fancyhf{}
\fancyhead[L]{\rightmark}
\fancyhead[R]{\thepage}
\renewcommand{\headrulewidth}{0.2pt}
\pagestyle{fancy}
\renewcommand{\headrulewidth}{0.2pt}

\begin{titlepage}
  \begin{figure}[t]
      \centering
      \includegraphics[width=0.7\textwidth]{drawings/logo_sapienza_new.png}
      \label{fig:logo}
  \end{figure}    
  \null\vfill
  \begin{center}
      \rule{\textwidth}{0.4mm} % Line above the title
      \\[0.8cm] % Space between line and title
      \begin{minipage}{\textwidth}
          \centering
          {\Huge \textbf{Cryptography}} 
      \end{minipage}
      \\[0.8cm] % Space between title and line
      \rule{\textwidth}{0.4mm} % Line below the title
      \vspace{2cm}
      { \\ \Large Colacel Alexandru Andrei}
  \end{center}
  \vfill\null
  \renewcommand{\abstractname}{Disclaimer}
  \begin{center}
      These notes are derived from the lectures of Prof. Daniele Venturi, from books, and other educational materials. The sale of this material is prohibited.
  \end{center}

\end{titlepage}

\tableofcontents


\chapter{Introduction}

\begin{quote}
	\textit{Solomon,\\ I'm concerned about security; I think, when we email each other, we should use some sort of code.}
\end{quote}

Confidentiality is our goal.
We want to encrypt and decrypt a (plaintext)\footnote{Plaintext usually means unencrypted information pending input into cryptographic algorithms, usually encryption algorithms.} message $m$, using a key, to obtain a cyphertext $c$.
As per Kirkoff's principle, only the key is secret.

\begin{figure}[h]
	\centering
	\includegraphics[width=0.8\linewidth]{drawings/symmetric-encryption.eps}
	\caption{Message exchange between $A$ and $B$ using symmetric encryption. $E$ is the eavesdropper.}
	\label{fig:symmetric-encryption}
\end{figure}

Our encryption schemes have the following syntax:
\begin{equation*}
	\GenericEncSchemeTuple.
\end{equation*}
$A$ and $B$, the actors of our communication exchange (\cref{fig:symmetric-encryption}), share $k$, the key, taken from some key space $\K$.
The elements of our encryption scheme play the following roles:
\begin{enumerate}
	\item $\GenericKeyGen$ outputs a random key from the key space $\K$, and we write this as $k \leftarrow \GenericKeyGen(1^n)$ where $n$ will quantify the security of the cipher;
	\item $\GenericEnc : \K \times \M \to \C$ is the encryption function, mapping a key and a message to a cyphertext\footnote{In cryptography, ciphertext or cyphertext is the result of encryption performed on plaintext using an algorithm, called a cipher. Ciphertext is also known as encrypted or encoded information because it contains a form of the original plaintext that is unreadable by a human or computer without the proper cipher to decrypt it.}. Given a message to be encrypted $m \in \M$ and the shared key $k$ calculate the cryptotext $c = \GenericEnc(m)$;
	\item $\GenericDec : \K \times \C \to \M$ is the decryption function, mapping a key and a cyphertext to a message. Given a cryptotext $c$ and the shared key $k$ returns the message $m \in \M$ such that $m = \GenericDec(c)$.
\end{enumerate}
We expect an encryption scheme to be at least correct and for each $n$, for each key $k \leftarrow \GenericKeyGen(1^n)$ and for every message $m \in \M$ we have:
\begin{equation*}
	\forall k \in \K, \forall m \in \M . \GenericDec(k, \GenericEnc(k, m)) = m.
\end{equation*}
An encryption scheme is defined by three algorithms $\GenericKeyGen$, $\GenericEnc$, and $\GenericDec$, as well as a specification of a message space $\M$ with $| \M | > 1$.\footnote{If $| M | = 1$ there is only one message and there is no point in communicating, let alone encrypting.}
\section{Perfect secrecy}

Shannon defined ``perfect secrecy'', \ie we will require that the adversary is not able to obtain any additional information about $m$. Formally we can express this request through probability theory.
\begin{definition}[Perfect secrecy]\label{def:perfect-secrecy}
	Let $M$ be a \ac{RV} over $\M$, and $K$ be a uniform distribution over $\K$.

	$(\GenericEnc, \GenericDec)$ has perfect secrecy if, $\forall m \in \M$ and $\forall c \in \C$ such that $\Pr {C = c} \geq 0$ \footnote{This requirement is necessary only to avoid influencing an event with zero probability. In the following we will always assume that $\Pr {M = m}$ and $\Pr {C = c}$ are positive. What has been shown can be generalized to the case of arbitrary distributions on $\M$ and $\C$.}:
	\begin{equation*}
		\forall M, \forall m \in \M, c \in \C . \Pr{M = m} = \Pr{M = m | C = c}
	\end{equation*}
	where $C = \GenericEnc(k,m)$ is a third \ac{RV}.
\end{definition}

We have equivalent definitions for perfect secrecy.
\begin{theorem}\label{thm:perfect-secrecy:equivalent-definitions}
	The following definitions are equivalent:
	\begin{enumerate}
		\item \label{itm:thm:perfect-secrecy:original} \cref{def:perfect-secrecy}: $\forall m \in \M$ and $\forall c \in \C$:
		\begin{equation*}
			\Pr{M = m} = \Pr{M = m | C = c}
		\end{equation*}
		\item \label{itm:thm:perfect-secrecy:independent} $\forall m \in \M$ and $\forall c \in \C$ $M$ and $C$ are independent;
		\item \label{itm:thm:perfect-secrecy:invariant}
		$\forall m, m' \in \M, \forall c \in \C \quad \footnote{The encryption algorithm may be probabilistic, so that $\GenericEnc(m)$ might output a different ciphertext when run multiple times. To emphasize this, we write $c \leftarrow \GenericEnc(m)$ to denote the possibly probabilistic process by which message $m$ is encrypted using key $k$ to give ciphertext $c$.}:$
			\begin{equation*}
				\Pr{\GenericEnc(k,m) = c} = \Pr{\GenericEnc(k,m') = c}
			\end{equation*}
			where $k$ is a random key in $\K$ chosen with uniform probability. \qedhere
	\end{enumerate}
\end{theorem}

It is sufficient to show that $(1) \Rightarrow (2) \Rightarrow (3) \Rightarrow (1)$




\begin{tcolorbox}[colframe=black, colback=sfondo_gray, sharp corners]
	
We can express the relationship between joint and marginal densities through the concept of \textit{conditional discrete density}. The probability of event $\mathcal{E}_1$ conditional on event $\mathcal{E}_2$ is the probability that event $\mathcal{E}_1$ occurs given that $\mathcal{E}_2$ occurs. For example, returning to the example of the die, the probability of getting a $\spadesuit$ knowing that the result of the roll will be $\spadesuit$ or $\heartsuit$ is $\frac{1}{3}$. Formally, we have:

\[
\Pr{\mathcal{E}_1 \mid \mathcal{E}_2} = \frac{\Pr{\mathcal{E}_1 \land \mathcal{E}_2}}{\Pr{\mathcal{E}_2}}.
\]

Two events are said to be \textit{statistically independent} when $\Pr{\mathcal{E}_1 \mid \mathcal{E}_2} = \Pr{\mathcal{E}_1}$ (i.e., when the probability that $\mathcal{E}_1$ occurs given that $\mathcal{E}_2$ occurs is equal to the probability that $\mathcal{E}_1$ occurs a priori). The relationship between $\Pr{\mathcal{E}_1 \mid \mathcal{E}_2}$ and $\Pr{\mathcal{E}_2 \mid \mathcal{E}_1}$ is given by Bayes' theorem, expressed by the following equation:

\[
\Pr{\mathcal{E}_2 \mid \mathcal{E}_1} = \Pr{\mathcal{E}_1 \mid \mathcal{E}_2} \cdot \frac{\Pr{\mathcal{E}_2}}{\Pr{\mathcal{E}_1}} \tag{A.3}
\]

Switching to densities, given two random variables $X \in \mathcal{X}$ and $Y \in \mathcal{Y}$, the conditional probability of $Y = y$ given $X = x$ is:
\begin{proof}
\[
\Pr{Y = y \mid X = x} = \frac{\Pr{X = x \land Y = y}}{\Pr{X = x}} \stackrel{\text{(A.3)}}{=} \frac{\Pr{Y = y \mid X = x} \Pr{Y = y}}{\Pr{X = x}}
\]
\end{proof}
\end{tcolorbox}


\begin{proof}[Proof of \cref{thm:perfect-secrecy:equivalent-definitions}]
	First, we show that \ref{itm:thm:perfect-secrecy:original} $\implies$ \ref{itm:thm:perfect-secrecy:independent}.
	\begin{align*}
		\Pr{M=m} & = \Pr{M = m | C = c} \\
		& = \frac{\Pr{M = m \land C = c}}{\Pr{C = c}} \tag{by Bayes}
		\\
		& \implies \\
		\Pr{M = m} \Pr{C = c}
		& =
		\Pr{M = m \land C = c}
	\end{align*}
	which is the definition of independence \footnote{In the context of probability theory: Both \( \land \) and \( \cap \) serve the purpose of indicating the intersection of events. The choice between \( \land \) and \( \cap \) may vary based on personal preference or the context in which it's used, but they convey the same meaning in probability theory.}.

	Now we show that \ref{itm:thm:perfect-secrecy:independent} $\implies$ \ref{itm:thm:perfect-secrecy:invariant}.
	\begin{align*}
		\Pr{\GenericEnc(k, m) = c} 
		& =
		\Pr{\GenericEnc(k, M) = c | M = m} \tag{we fixed $m$}
		\\
		& = \Pr{C = c | M = m} \tag{definition of the \ac{RV} $C$}
		\\
		& = \Pr{C = c}. \tag{by \ref{itm:thm:perfect-secrecy:independent}}
	\end{align*}
	Since $m$ is arbitrary, we can do the same for $m'$, and obtain
	\begin{equation*}
		\Pr{\GenericEnc(k,m') = c} = \Pr{C = c}
	\end{equation*}
	which gives us \ref{itm:thm:perfect-secrecy:invariant}.

	Now we want to show that \ref{itm:thm:perfect-secrecy:invariant} $\implies$ \ref{itm:thm:perfect-secrecy:original}. Assume that the encryption scheme is perfectly secret and fix messages $m$, $m\prime$ $\in \M$ and a ciphertext $c \in \C$. Take any $c \in \C$. By \ref{itm:thm:perfect-secrecy:independent} we have:

	\begin{equation*}
		\Pr {C = c | M = m} = \Pr {C = c} = \Pr {C = c | M = m'}.
	\end{equation*}

	completing the proof of the first direction.
	Assume next that for every distribution over $\M$, every $m$, $m\prime \in \M$, and every $c \in \C$ it holds that $\Pr {C = c | M = m} = \Pr {C = c | M = m\prime}$. Fix some distribution over $\M$, and an arbitrary $m \in \M$ and $c \in \C$. Define $p \stackrel{\text{def}}{=}$ $\Pr {C = c | M = m}$. Since $\Pr {C = c | M = m} = \Pr {C = c | M = m'} = p$ for all $m$, we have:

	\begin{align*}
		\Pr{C = c}
		& =
		\sum_{m' \in \M} \Pr{C = c \land M = m'}
		\\
		& = 
		\sum_{m' \in \M} \Pr{C = c | M = m'} \Pr{M = m'}
		\tag{by Bayes}
		\\
		& =
		\sum_{m' \in \M} \Pr{\GenericEnc(k,M) = c | M = m'} \Pr{M = m'}
		\\
		& =
		\sum_{m' \in \M} \Pr{\GenericEnc(k,m') = c} \Pr{M = m'}
		\\
		& =
		\Pr{\GenericEnc(k, m) = c} \underbrace{\sum_{m' \in \M} \Pr{M = m'}}_{1}
		\tag{$\GenericEnc$ is indepenendent of $M$, so we take it out}
		\\
		& =
		\Pr{\GenericEnc(k,M) = c | M = m} =
		\Pr{C = c | M = m}.
	\end{align*}
	We are left to show that $\Pr{M = m} = \Pr{M = m | C = c}$, but this is easy with Bayes.
\end{proof}


\begin{theorem}(Shannon Theorem)
	Let $|\M| = |\K| = |\C|$. A cipher has perfect secrecy if and only if:
	\begin{enumerate}[label=(\roman*)]
		\item Each key $k \in \K$ is chosen uniformly with probability $\frac{1}{|\K|}$;
		\item For every message $m \in \M$ and every ciphertext $c \in \C$, there exists one and only one key $k$ such that $c = \GenericKeyGen(m)$.
	\end{enumerate}
\end{theorem}

\section{\acl{OTP} (Vernam's Cipher)}

Now we'll see a perfect encryption scheme, the \ac{OTP}.



\begin{tcolorbox}[colframe=black, colback=sfondo_gray, sharp corners]

\begin{construction}[\acl{OTP}]
	The message space, the cyphertext space, and the key space are all the same, \ie $\M = \K = \C = \{0,1\}^{l}$, with $l \in \Naturals$.

	The \acl{OTP} encryption scheme by Vernam is defined as follows:
	
	\begin{itemize}
		\item Fix an integer $l$ $>$ 0. Then the message space $\M$, key space $\K$, and ciphertext space $\C$ are all equal to $\{0, 1\}^l$ (i.e., the set of all binary strings of length $l$);
		\item The key-generation algorithm $\GenericKeyGen$ works by choosing a string from $\K = \{0, 1\}^l$ according to the uniform distribution (i.e., each of the $2^l$ strings in the space is chosen as the key with probability exactly $2^{-l}$). Generate a random key $k \overset{\$}{\leftarrow} \{0,1\}^l$;
		\item $\GenericEnc$: to encrypt the message \( m \in \{0, 1\}^l \), we define:
		\begin{equation*}
			c = \text{Enc}(m) = k \oplus m
		\end{equation*}
		\item $\GenericDec$: To decrypt the ciphertext $c \in \{0, 1\}^l$ with key $k$ compute:
		\begin{equation*}
			m = \text{Dec}(c) = c \oplus k
		\end{equation*}
	\end{itemize}
\end{construction}
\end{tcolorbox}
Seeing that this is correct is immediate.

This can actually be done in any finite abelian\footnote{An abelian group, also known as a commutative group, is a fundamental concept in abstract algebra where the group's binary operation (commonly denoted as \(+\) or multiplication in different contexts) is commutative. This means that the order of elements in the operation does not affect the result. Examples include additive groups of integers (\(\mathbb{Z}\)), rational numbers (\(\mathbb{Q}\)), and multiplicative groups of non-zero rational numbers (\(\mathbb{Q}^*\)), as well as matrix groups with matrix multiplication. Abelian groups play a crucial role in abstract algebra, with many theorems and concepts specifically applying to them.
} group $(\Group, +)$, where you just do $k + m$ to encode and $c - k$ to decode.


\begin{theorem} \label{thm:otp:perfect-secrecy}
	\ac{OTP} is perfectly secure.
\end{theorem}

\begin{proof}[Proof of \cref{thm:otp:perfect-secrecy}]
	Fix $m \in \M, c \in \C$, and choose a random key.
	\begin{equation*}
		\Pr{\OTPEnc(k,m) = c} = \Pr{k = c - m} = \frac{1}{\abs{\Group}}.
	\end{equation*}
	This is true for any $m$, so we are done.
\end{proof}

\ac{OTP} has two problems:
\begin{enumerate}
	\item the key is long (as long as the message);
	\item we can't reuse the key:
	\begin{equation*}
		\begin{array}{c}
			c = k + m \\
			c' = k + m'
		\end{array}
		\implies
		k = c - m = c' - m'
		\implies
		m' = m - (c - c').
	\end{equation*}
\end{enumerate}

\begin{theorem}[Shannon, 1949] \label{thm:shannon:1949}
	In any perfectly secure encryption scheme the size of the key space is at least as large as the size of the message space, \ie $\abs{\K} \ge \abs{\M}$.
\end{theorem}

\begin{proof}[Proof of \cref{thm:shannon:1949}]
	Assume, for the sake of contradiction, that $\abs{\K} < \abs{\M}$.
	Fix $M$ to be the uniform distribution over $\M$, which we can do as perfect secrecy works for any distribution.
	Take a cyphertext $c \in \C$ such that $\Pr{C = c} > 0$, \ie $\exists m, k$ such that $\OTPEnc(k,m) = c$.

	Consider $\M' = \{ \OTPDec(k,c) : k \in \K \}$, the set of all messages decrypted from $c$ using any key.
	Clearly, $\abs{\M'} \le \abs{\K} < \abs{\M}$, so $\exists m' \in \M$ such that $m' \not\in \M'$.
	This means that
	\begin{equation*}
		\Pr{M = m'} = \frac{1}{\abs{\M}} \neq \Pr{M = m' | C = c} = 0
	\end{equation*}
	in contradiction with perfect secrecy.
\end{proof}

In the rest of the course we will forget about perfect secrecy, and strive for computational security, \ie bound the computational power of the adversary.

\section{Authentication}

The aim of authentication is to avoid tampering of $E$ with the messages exchanged between $A$ and $B$ (\cref{fig:symmetric-authentication}).

\begin{figure}
	\centering
	\includegraphics[width=0.8\linewidth]{drawings/symmetric-authentication.eps}
	\caption{Message exchange between $A$ and $B$ using symmetric authentication. $E$ is the eavesdropper.}
	\label{fig:symmetric-authentication}
\end{figure}

A \ac{MAC} is defined as a tuple $\GenericMACSchemeTuple$, where:
\begin{itemize}
	\item $\GenericKeyGen$, as usual, outputs a random key from some key space $\K$;
	\item $\GenericMac : \K \times \M \to \GenericAuthSpace$ maps a key and a message to an authenticator in some authenticator space $\GenericAuthSpace$;
	\item $\GenericVrfy : \K \times \M \times \GenericAuthSpace \to \Bool$ verifies the authenticator.
\end{itemize}
As usual, we expect a \ac{MAC} to be correct, \ie
\begin{equation*}
	\forall m \in \M, \forall k \in \K . \GenericVrfy(k, m, \GenericMac(k, m)) = 1.
\end{equation*}

If the $\GenericMac$ function is deterministic, then it must be that $\GenericVrfy(k, m, \phi) = 1$ if and only if $\GenericMac(k, m) = \phi$.

Security for \acp{MAC} is that \emph{forgery} must be hard: you can't come up with an authenticator for a message if you don't know the key.

\begin{definition}[Information theoretic \acs{MAC}]
	$(\GenericMac, \GenericVrfy)$ has $\epsilon$-statistical security if for all (possibly unbounded) adversary $\Adv$, for all $m \in \M$,
	\begin{equation*}
		\Pr{
			\GenericVrfy(k, m', \phi') = 1 \land m' \neq m :
			\begin{array}{l}
				k \rand{\KeyGen}; \; \\
				\phi = \GenericMac(k,m); \; \\
				(m', \phi') \from \Adv(m,\phi)
			\end{array}
		} \le \epsilon
	\end{equation*}
	\ie the adversary forges a $(m',\phi')$ that verifies with key $k$ with low probability, even if it knows a valid pair $(m, \phi)$.
\end{definition}

As an exercise, prove that the above is impossible if $\epsilon = 0$.

Information theoretic security is also called unconditional security.
Later we'll see \emph{conditional} security, based on computational assumptions.

\begin{definition}[Pairwise independence]
	Given a family $\H = \{ h_k : \M \to \GenericAuthSpace \}_{k \in \K}$ of functions, we say that $\H$ is pairwise independent if for all distinct $m, m'$ we have that $(h_k(m), h_k(m')) \in \GenericAuthSpace^2$ is uniform over the choice of $k \rand{\K}$.
\end{definition}

We show straight away a construction of a pairwise independent family of function.
\begin{construction}[Pairwise independent function] \label{cons:pairwise-independent}
	Let $p$ be a prime, the functions in our family $\H$ are defined as
	\begin{equation*}
		h_{a,b}(m) = a m + b \mod p
	\end{equation*}
	with $\K = \Integers^2_p$, and with $\M = \GenericAuthSpace = \Integers_p$.
\end{construction}

\begin{theorem} \label{thm:mod-pairwise-independent}
	\Cref{cons:pairwise-independent} is pairwise independent.
\end{theorem}

\begin{proof}[Proof of \cref{thm:mod-pairwise-independent}]
	For any $m, m'$, $\phi, \phi'$, we want to find the value of
	\begin{equation*}
		\Pr{a m + b = \phi \land a m' + b = \phi'}
	\end{equation*}
	for $a,b \rand{\Integers^2_p}$.
	This is the same as
	\begin{equation*}
		\Pr[a,b]{
			\begin{pmatrix}
				m & 1 \\
				m' & 1
			\end{pmatrix}
			\begin{pmatrix}
			a \\ b
			\end{pmatrix}
			=
			\begin{pmatrix}
			\phi \\ \phi'
			\end{pmatrix}
		}
		=
		\Pr[a,b]{
			\begin{pmatrix}
			a \\ b
			\end{pmatrix}
			=
			{
			\begin{pmatrix}
				m & 1 \\
				m' & 1
			\end{pmatrix}
			}^{-1}
			\begin{pmatrix}
			\phi \\ \phi'
			\end{pmatrix}
		}
		=
		\frac{1}{\abs{\GenericAuthSpace}^2}.
	\end{equation*}
	This is true since 
	${\begin{psmallmatrix}
		m & 1 \\
		m' & 1
	\end{psmallmatrix}
	}^{-1}
	\begin{psmallmatrix}
	\phi \\ \phi'
	\end{psmallmatrix}$
	is just a couple of (constant) numbers, so the probability of choosing $(a,b)$ such that they equal the constant is just $\frac{1}{\abs{\GenericAuthSpace}^2}$.
\end{proof}

If $h_k$ is part of a pairwise independent family of functions, then $\GenericMac(k,m) = h_k(m)$, and $\GenericVrfy(k,m,\phi)$ is simply computing $h_k(m)$ and comparing it with $\phi$, \ie
\begin{equation*}
	\GenericVrfy(k,m,\phi) = 1 \iff h_k(m) = \phi.
\end{equation*}
We now prove that this is an information theoretic \ac{MAC}.

\begin{theorem} \label{thm:pairwise-independent:statistical-security}
	Any pairwise independent function is $\frac{1}{\abs{\GenericAuthSpace}}$-statistical secure.
\end{theorem}

\begin{proof}[Proof of \cref{thm:pairwise-independent:statistical-security}]
	Take any two distinct $m, m'$, and two $\phi, \phi'$.
	We show that the probability that $\GenericMac(k,m') = \phi'$ is exponentially small.
	\begin{equation*}
		\Pr[k]{\GenericMac(k,m) = \phi} =
		\Pr[k]{h_k(m) = \phi} =
		\frac{1}{\abs{\GenericAuthSpace}}.
	\end{equation*}
	Now look at the joint probabilities:
	\begin{align*}
		\Pr[k]{\GenericMac(k,m) = \phi \land \GenericMac(k,m') = \phi'} 
		& = 
		\Pr[k]{h_k(m) = \phi \land h_k(m') = \phi'} 
		\tag{by definition}
		\\
		& =
		\frac{1}{\abs{\GenericAuthSpace}^2}
		=
		\frac{1}{\abs{\GenericAuthSpace}}
		\cdot
		\frac{1}{\abs{\GenericAuthSpace}}.
	\end{align*}
	The last steps come from the fact that $h_k$ is pairwise independent.
	To see that the construction is $\frac{1}{\abs{\GenericAuthSpace}}$-statistical secure:
	\begin{align*}
		\Pr[k]{\GenericMac(k, m') = \phi' | \GenericMac(k,m) = \phi}
		& =
		\Pr[k]{h_k(m') = \phi' | h_k(m) = \phi}
		\\
		& =
		\frac{\Pr[k]{h_k(m) = \phi \land h_k(m') = \phi'}}{\Pr[k]{h_k(m) = \phi}}
		\\
		& =
		\frac{1}{\abs{\GenericAuthSpace}}.
	\end{align*}
\end{proof}

Note that \cref{cons:pairwise-independent} ($h_k(m) = a m + b \mod p$) is insecure if the same key $k = (a,b)$ is used for two messages.

\begin{theorem}
	Any $t$-time $2^{-\lambda}$-statistically secure \ac{MAC} has key of size $(t+1)\lambda$.
\end{theorem}

\section{Randomness Extraction}

$X$ is a random source (possibly not uniform).
$\Ext(X) = Y$ is a uniform \ac{RV}.

First, let's see a construction for a binary \ac{RV}.
Let $B$ be a \ac{RV} such that $\Pr{B = 1} = p$ and $\Pr{B = 0} = 1-p$, with $p \neq 1-p$.
We take two samples, $B_1$ and $B_2$ from $B$, and we want to obtain an unbiased \ac{RV} $B'$.
\begin{enumerate}
	\item Take two samples, $b_1 \rand{B_1}$ and $b_2 \rand{B_2}$;
	\item if $b_1 = b_2$, sample again;
	\item if $(b_1 = 1 \land b_2 = 0)$, output 1; if $(b_1 = 0 \land b_2 = 1)$, output 0.
\end{enumerate}
It's easy to verify that $B'$ is uniform:
\begin{align*}
	\Pr{B' = 1} & = \Pr{B_1 = 1 \land B_2 = 0} = p (1-p) \\
	\Pr{B' = 0} & = \Pr{B_1 = 0 \land B_2 = 1} = (1-p) p.
\end{align*}
How many trials do we have to make before outputting something?
$2(1-p)p$ is the probability that we output something.
The probability that we don't output anything for $n$ steps is thus $(1 - 2(1-p)p)^n$.


% missing some stuff on randomness extraction

\pagebreak
\acresetall

\input{computational-crypto.tex}
\pagebreak

\acresetall

\input{ske.tex}
\pagebreak

\acresetall

\input{number-theory.tex}
\pagebreak

\acresetall

\input{pke.tex}
\pagebreak

\acresetall

\input{signature-schemes.tex}
\pagebreak

\acresetall

\input{acronyms.tex}

\end{document}
