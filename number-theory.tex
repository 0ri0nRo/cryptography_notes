
\section{Number Theory}

We use modular arithmetic, with modulus some $n$, in $(\Integers_n, +, \cdot)$.

$(\Integers_n, +)$ is a group, \ie it has an identity (or null) element, it's closed, commutative, associative, and has an inverse for each element.
$(\Integers_n, \cdot)$ is not a group, as you don't have an inverse for everyone.

\begin{lemma} \label{lem:gcd-invertible}
	If $\gcd(a,n) > 1$, $a$ is not invertible in $(\Integers_n, \cdot)$.
\end{lemma}

\begin{proof}[Proof of \cref{lem:gcd-invertible}]
	We can write $a = b \cdot q + (a \mod b)$.
	Assume $a$ is invertible, then $a \cdot b = q \cdot n + 1$.
	But then $\gcd(a,n)$ divides $a \cdot b - q \cdot n = 1$, which is a contradiction.
\end{proof}

Operations in $\Integers_n$ (at least multiplication and addition) are efficient.
The inverse of an element can be computed with the euclidean algorithm.

\begin{lemma} \label{lem:euclidean-algorithm}
	Let $a,b$ such that $a \ge b > 0$, then
	\begin{equation*}
		\gcd(a,b) = \gcd(b, a \mod b). \qedhere
	\end{equation*}
\end{lemma}

\begin{proof}[Proof of \cref{lem:euclidean-algorithm}]
	It suffices to show that a common divisor of $a$ and $b$ also divides $a \mod b$, and that a common divisor of $b$ and $a \mod b$ also divides $a$.

	Write $a = q \cdot b + a \mod b$.

	For the first implication, a common divisor of $a$ and $b$ divides $a - q \cdot b = a \mod b$.

	For the second implication, a common divisor of $b$ and $a \mod b$ divides $b \cdot q + a \mod b = a$.
\end{proof}

\begin{theorem} \label{thm:euclidean-algorithm-poly-time}
	Given $a,b$ we can compute $\gcd(a,b)$ in polynomial time in $\max \{ \norm{a}, \norm{b} \}$.
	Also, we can find $u, v$ such that $a \cdot u + b \cdot v = \gcd(a,b)$.
\end{theorem}

\begin{proof}[Proof of \cref{thm:euclidean-algorithm-poly-time}]
	We can use \cref{lem:euclidean-algorithm} recursively:
	\begin{align*}
		a & = b \cdot q_{1} + r_{1} \tag{$0 \le r_1 < b$} \\
		\gcd(a,b) & = \gcd(b, r_1) \tag{$r_1 = a \mod b$} \\
		b & = r_{1} \cdot q_{2} + r_{2} \tag{$0 \le r_2 < r_1$} \\
		\gcd(b,r_1) & = \gcd(r_1, r_2) \tag{$r_2 = b \mod r_1$} \\
		& \dots \\
		r_{i} & = r_{i-1} \cdot q_{i+1} + r_{i+1}.
	\end{align*}
	Repeat until $r_{t+1} = 0$, we then have that $\gcd(a,b) = r_t$.

	We prove that $r_{i+2} \le \frac{r_i}{2}$ for all $0 \le i \le t-2$.
	Clearly, $r_{i+1} < r_{i}$.
	If we have that $r_{i+1} \le \frac{r_i}{2}$, then it's trivial.
	If $r_{i+1} > \frac{r_i}{2}$, then
	\begin{equation*}
		r_{i+2} = r_i \mod r_{i+1} = r_i - q_{i+2} r_{i+1}
		< r_i - r_{i-1} < r_i - \frac{r_i}{2} = \frac{r_i}{2}. \qedhere
	\end{equation*}
\end{proof}

Take $a \in \Integers_n$ such that $\gcd(a,n) = 1$, then there are $u,v$ such that $a \cdot u + n \cdot v = \gcd(a,n) = 1$.
But then $a \cdot u \equiv 1 \mod n$, so $u$ is the inverse of $a$.

Another operation in $\Integers_n$ is modular exponentiation, \ie $a^b \mod n$.
The square and multiply algorithm is polynomial.

Write $b$ in binary, as $b_t b_{t-1} \dots b_0$, so $b \in \{0,1\}^{t+1}$.
Since $b = \sum_{i = 0}^{t} b_i \cdot 2^i$, we can write
\begin{align*}
	a^b = a^{\sum_{i = 0}^{t} 2^i b_i} =
	\prod_{i = 0}^{t} a^{2^i b_i} =
	\prod_{i = 0}^{t} \left( a^{2^i} \right)^{b_i} =
	a^{b_0} \left( a^2 \right)^{b_1} \left( a^4 \right)^{b_2} \dots \left( a^{2^t} \right)^{b_t}.
\end{align*}
We have $t$ multiplications and $t$ squares, so modular exponentiation happens in polynomial time.

\begin{theorem}[Prime Number Theorem] \label{thm:prime-number-theorem}
	\begin{equation*}
		\Pi(x) = \text{``\# of primes $\le x$''} \ge \frac{x}{3 \log_2(x)}
	\end{equation*}
	which is roughly $\frac{x}{\log(x)}$.
\end{theorem}

\Cref{thm:prime-number-theorem} means that
\begin{equation*}
	\Pr{x \text{ is prime} : x \rand{2^{\lambda - 1}}} \ge \frac{\frac{2^{\lambda} - 1}{3 \log_2(2^{\lambda} - 1)}}{2^{\lambda - 1}} \approx \frac{1}{3\lambda}.
\end{equation*}
Furthermore, primality testing can be done efficiently.

\begin{theorem}[Miller-Rabin '80s, Agramal-Kayer-Saxana '02]
	You can test in polynomial time if $x$ is prime.
\end{theorem}

We can sample $x \rand{2^{\lambda} - 1}$, test primality, and eventually resample.
\begin{equation*}
	\Pr{\text{no output after $t$ samples}} \le \left( 1 - \frac{1}{3\lambda}\right)^{t} \le \left(\frac{1}{e} \right)^{\lambda} \in \negl{\lambda}
\end{equation*}
for $t = 3 \lambda^2$, since $\left( 1 - \frac{1}{x} \right)^x \le \frac{1}{e}$.

Now, for some new number theoretic assumptions.

\paragraph{Assumption.} Integer factorisation of the product of two $\lambda$-bit primes is a \ac{OWF}.
\ac{NIST} recommendation is $\lambda \approx 2048$.

We need cyclic groups.

\begin{theorem}[Lagrange] \label{thm:lagrange}
	If $H$ is a subgroup of $G$, then $\abs{H} \divides \abs{G}$.
\end{theorem}

We define the order of $a$ to b e the minimum $i$ such that $a^i \equiv 1 \mod n$.

Consider $\Integers_n$.
Note that $\abs{\Integers_n} = n$.
We define
\begin{equation*}
	\Integers_n^{\star} = \{ a \in \Integers_n : a \text{ is invertible}\} =
	\{ a \in \Integers_n : \gcd(a,n) = 1 \}.
\end{equation*}
$\abs{\Integers_n^{\star}} \Definition \varphi(n)$.
If $n$ is a prime $p$, then $\Integers_p^{\star} = \{ 1, 2, \dots, p - 1\}$, and $\abs{\Integers_p^{\star}} = \varphi(p) = p-1$.

\begin{corollary}
	For all $a \in \Integers_n^{\star}$, we have
	\begin{enumerate}
		\item $a^b = a^{b \mod \varphi(n)} \mod n$;
		\item $a^{\varphi(n)} = 1 \mod n$;
		\item $a^{p-1} = 1 \mod p$. \qedhere
	\end{enumerate}
\end{corollary}

$(\Integers_p^{\star}, \cdot)$ is cyclic, \ie $\exists$ a generator $g$ such that
\begin{equation*}
	\Integers_p^{\star} = \Generator{g} = \{ g^0, g^1, \dots, g^{p-2} \}.
\end{equation*}
Can we find a generator for $\Integers_p^{\star}$?
We can do it if we know a factorisation of $p-1 = \prod_{i=1}^{t} p_i^{\alpha_i}$.
Since $p_i \ge 2$, we have that $2^t \le p < 2^{\lambda+1}$, thus $t \le \lambda$.
The algorithm requires $t$ steps.

By \cref{thm:lagrange} we know that the order of any $y \in \Integers_p^{\star}$ divides $p-1$.
Thus $y$ is not a generator $\iff \exists 1 \le i \le t$ such that
\begin{equation*}
	y^{\frac{p-1}{p_i}} \equiv q \mod p.
\end{equation*}
If this is the case, $y$ is a generator of a subgroup of $\Integers_p^{\star}$.
This leads us to the next computational assumption.

\paragraph{Assumption}
The \ac{DL} problem, that is finding $x$ given $g^x \in \Group$, is hard, \ie for all \ac{PPT} $\Adv$,
\begin{equation*}
	\Pr{
		y = g^{x'} :
		\begin{array}{l}
			\GroupGen; x \rand{\Integers_q}; \\
			y = g^x; x' \from \Adv(y, \Group, g, q, 1^{\lambda})
		\end{array}
	}
	\le \epsilon(\lambda).
\end{equation*} 

\subsection{Elliptic Curves}

$\Group$ is a group of points over some cubic curve modulo $p$, \ie
\begin{equation*}
	y = x^3 + a x^2 + b x + c \mod p.
\end{equation*}
We can define an operation $+$ that makes $\Group$ a group.

$\GroupGen$, and take as an example $\Group = \Integers_p^{\star}$ with $\cdot$ operation (multiplication).
$q = p-1$.
Take $y \in \Integers_p^{\star}$, we have that $y = g^x$ for $x \in \Integers_{p-1}$.

In the general case, we have $y \in \Group$, and that $y = g^x$ for $x \in \Integers_q$.
$q$ is the order of the group, while $g$ is a generator for $\Group$.
Since $y = g^x$ we have that $\log_g(y) = x$ in $\Group$.

$\Integers_p^{\star}$ is a special case, since modular exponentiation is a \ac{OWP}.

\begin{definition}[\acl{CDH}]
	Intuitively, given $(g, g^x, g^y)$, it's hard to find $g^{xy}$.
	\ac{CDH} holds in $\Group$ if $\forall$ \ac{PPT} $\Adv$ we have
	\begin{equation*}
		\Pr{
			z = g^{xy} :
			\begin{array}{l}
				\params = \GroupTuple ; x, y \rand{\Integers_q} ; \\
				z \from \Adv(\params, g^x, g^y)
			\end{array}
		}
		\le \negl{\lambda}. \qedhere
	\end{equation*}
\end{definition}

\begin{observation} \label{obs:cdh-dl}
	\ac{CDH} $\implies$ \ac{DL}.
\end{observation}

\begin{proof}[Proof of \cref{obs:cdh-dl}]
	Assume not, then \ac{DL} does not hold if \ac{CDH} holds.
	But if we can compute $y = \log_g(g^y)$, then it's easy to find $g^{xy} = (g^x)^y$.
\end{proof}
Whether \ac{DL} $\implies$ \ac{CDH} or not is not known.

\begin{definition}[\acl{DDH}]
	Intuitively,
	\begin{equation*}
		\underbrace{(g, g^x, g^y, g^{xy})}_{\text{\acs{DDH} tuple}} \CompInd \underbrace{(g, g^x, g^y, g^z)}_{\text{non-\acs{DDH} tuple}}.
	\end{equation*}
	\ac{DDH} holds in $\Group$ if for all \ac{PPT} $\Adv$
	\begin{align*}
		& \left|
			\Pr[x, y \rand{\Integers_q}]{\Adv(1^{\lambda}, \GroupTuple, g, g^x, g^y, g^{xy}) = 1}
		\right. \\
			&~-
		\left.
			\Pr[x, y,z \rand{\Integers_q}]{\Adv(1^{\lambda}, \GroupTuple, g, g^x, g^y, g^z) = 1}
		\right|
		\le \negl{\lambda}.
	\end{align*}
\end{definition}

\begin{observation}
	\ac{CDH} $\implies$ \ac{DDH} is true.
	\ac{DDH} $\implies$ \ac{CDH} maybe is not true.
\end{observation}

\begin{proposition} \label{prop:ddh-not-zp}
	\ac{DDH} does not hold in $\Integers_p^{\star}$.
\end{proposition}

\begin{proof}[Proof of \cref{prop:ddh-not-zp}]
	Consider the set
	\begin{equation*}
		\QR = \{ y \in \Integers_p^{\star} : y = x^2, x \in \Integers_p^{\star} \} = \{ y = g^z : z \text{ even} \}.
	\end{equation*}
	We can test if $y \in \QR$ by checking if $y^{\frac{p-1}{2}} \equiv 1 \mod p$, because if $y = g^{2z'}$ then $y^{\frac{p-1}{2}} \equiv \left( g^{p-1} \right)^{z'} \equiv 1 \mod p$.
	If not, then it must be that $y = 2 z' + 1$, and as such
	\begin{equation*}
		y^{\frac{p-1}{2}} \equiv \left( g^{p-1} \right)^{z'} \cdot g^{\frac{p-1}{2}} \equiv -1 \mod p.
	\end{equation*}

	Clearly,
	\begin{equation*}
		g^{xy} \in \QR \iff g^{x} \in \QR \lor g^y \in \QR.
	\end{equation*}
	Thus $g^{xy} \in \QR$ with probability $\frac{3}{4}$, but $g^z \in \QR$ only with probability $\frac{1}{2}$.
	So we have a distinguisher.

	$\Adv_{\text{DDH}}(g, g^x, g^y, g^z)$ outputs 1 if $g^x$ or $g^y$ are in $\QR$, but $g^z \not\in \QR$, and $0$ otherwise.
	$\Adv_{\text{DDH}}$ distinguishes with probability greater than $\frac{3}{8}$.
\end{proof}
This result can be improved by returning 0 if and only if $g^x$, $g^y$ and $g^z$ are compatible with $\QR$ considerations, obtaining a distinguisher operating with probability $\frac{1}{2}$.

If we take $\Group = \QR$, with $q = \frac{p-1}{2}$, and where both $q$ and $p$ are prime (Sophie Germain primes), maybe \ac{DDH} holds there.
This is not always true.

\subsection{\acl{DH} Key Exchange}

Take $\GroupGen$.
Alice samples $x \rand{\Integers_q}$, and sends $\X = g^x$ to Bob.
Bob samples $y \rand{\Integers_q}$, and sends $\Y = g^y$ to Alice.
Alice computes $\K_A = \Y^x$, Bob computes $\K_B = \X^y$.
It's easy to see that $\K_A = \K_B$.

\ac{CDH} implies that an adversary can't compute the key.
\ac{DDH} implies that the keys are indistinguishable from random.

This is not secure against man in the middle attacks.
Authentication would be needed, an initial fixed key.
% add drawing of man in the middle attack here

\subsection{\aclp{PRG}}

$\GroupGen$, $x,y \rand{\Integers_q}$.
From \ac{DDH} we know that
\begin{equation*}
	G_{g,q}(x,y) = (g^x, g^y, g^{xy}) \CompInd \U_{\Group^3}
\end{equation*}
so we have a \ac{PRG} $G_{g,q} : \Integers_q^2 \to \Group^3$.

\begin{construction}[\acs{PRG} from \acs{DDH}] \label{cons:ddh-prg}
	We can construct directly from \ac{DDH} a \ac{PRG} with polynomial stretch
	\begin{equation*}
		\G_{g,q} : \Integers_p^{l+1} : \Group^{2l+1}
	\end{equation*}
	defined as
	\begin{equation*}
		G_{g,q}(x, y_1, \dots, y_l) = 
		(g^{x}, g^{y_1}, g^{x y_1}, \dots, g^{y_l}, g^{x y_l}). \qedhere
	\end{equation*}
\end{construction}

\begin{theorem} \label{thm:ddh-prg}
	If \ac{DDH} holds, \cref{cons:ddh-prg} is a \ac{PRG}.
\end{theorem}

\begin{proof}[Proof of \cref{thm:ddh-prg}]
	Using the standard hybrid argument, since \ac{DDH} is $\epsilon$-hard, we would use $l$ hybrids, obtaining that the above \ac{PRG} is $(\epsilon \cdot l)$-secure.
	This proof is not tight, and we want to make a tight one.

	We make a direct reduction to \ac{DDH}.
	Let $\Adv_{\text{DDH}}$ be an adversary who is given $(g, g^x, g^y, g^z)$ with $z = \beta + xy$, where $\beta$ is either $0$ (so that is a \ac{DDH} tuple) or $\beta \rand{\Integers_q}$ (non-\ac{DDH} tuple).

	We want to prove that
	\begin{equation*}
		(g^{x}, g^{y_1}, g^{x y_1}, \dots, g^{y_l}, g^{x y_l})
		\CompInd
		(g^{x}, g^{y}, g^{z}, \dots)
	\end{equation*}
	by generating exactly the first if the tuple is \ac{DDH}, and exactly the other one if the tuple is not \ac{DDH}.

	For $j \in [l]$, pick $u_j, v_j \rand{\Integers_q}$, and define $g^{y_j} = \left( g^{y} \right)^{u_j} \cdot g^{v_j}$, and $g^{z_j} = \left( g^z \right)^{u_j} \cdot \left(g^x\right)^{v_j}$.
	$\Adv_{\text{DDH}}$ returns the same as
	\begin{equation*}
		\Adv_{\text{PRG}}(g^x, g^{y_1}, g^{z_1}, \dots, g^{y_l}, g^{z_l}).
	\end{equation*}

	This trick generates many tuples in the correct way: just look at the exponents.
	\begin{align*}
		y_j
		& = u_j \cdot y + v_j \\
		z_j
		& = u_j \cdot z + v_j \cdot x
		= u_j \cdot \beta + u_j \cdot x \cdot y + v_j \cdot x \\
		& = u_j \cdot \beta + x \cdot (u_j \cdot y + v_j)
		= u_j \cdot \beta + x \cdot y_j
	\end{align*}
	which is either sampled random from $\Integers_q$ if $\beta \rand{\Integers_q}$, or $x y_j$ if $\beta = 0$, with $y_j \rand{\Integers_q}$.
	So breaking the \ac{PRG} is equivalent to breaking \ac{DDH}, since $\Adv_{\text{DDH}}$ can perfectly simulate the \ac{PRG}, query $\Adv_{\text{PRG}}$ and break \ac{DDH}.
\end{proof}

\subsection{\acp{PRF}: Naor-Reingold}

\begin{construction}[Naor-Reingold]
	$\GroupGen$.
	\begin{equation*}
		\F_{\text{NR}} = \{ F_{q,g,\vec{a}} : \{0,1\}^{n} \to \Group \}_{\vec{a} \in \Integers_q^{n+1}}.
	\end{equation*}
	$\vec{a}$ is a vector of exponents.
	\begin{equation*}
		F_{q,g,\vec{a}}(x_1, \dots, x_n) = \left( g^{a_0} \right)^{\sum_{i=1}^{n} a_i x_i}. \qedhere
	\end{equation*}
\end{construction}
\ac{DDH} $\implies$ $\F_{\text{NR}}$ is a \ac{PRF}.
The proof can be sketched in a similar way as the proof for \ac{GGM}.
Note that this is like an optimised version of \ac{GGM}.

\subsection{Number Theoretic \acl{CFP}}

\begin{construction}[Number Theoretic \acs{CFP}] \label{cons:cfp-number-theory}
	\begin{enumerate}
		\item $\params = \GroupGen$;
		\item $y \rand{\Group}$, $\pk = (\params, y)$;
		\item define $f = (f_0, f_1)$, with:
			\begin{align*}
				f_0(\pk, x_0) & = g^{x_0}, \\
				f_1(\pk, x_1) & = y \cdot g^{x_1}.
			\end{align*}
	\end{enumerate}
\end{construction}
$\Group = \Integers_p^{\star}$, $q = p-1$, $\X_{\pk} = \Integers_q$.
Let $(x_0, x_1)$ be a claw, \ie $f_0(\pk, x_0) = f_1(\pk, x_1)$.
Then
\begin{equation*}
	g^{x_0} = y \cdot g^{x_1} \implies y = g^{x_0 - x_1} \implies x_0 - x_1 = \log_g(y).
\end{equation*}

\begin{theorem} \label{thm:cfp-number-theory}
	Under \ac{DL} \cref{cons:cfp-number-theory} is a \ac{CFP}.
\end{theorem}

\begin{proof}[Proof of \cref{thm:cfp-number-theory}]
	% add drawing
	It's a simple simulation.
\end{proof}

\begin{equation*}
	\H_{\pk}(x || m) = f_{m_l}( \dots f_{m_1} (x) \dots).
\end{equation*}
If $l = 1$, $\H_{\pk}(x || b) = f_b(\pk, m) = y^{b} g^x$.
Collision means $(x, b) \neq (x', b')$ such that $y^b g^x = y^{b'} g^{x'}$.
Clearly
\begin{equation*}
	b \neq b' \implies y^{b-b'} = g^{x'-x} \implies y = g^{(x'-x)(b-b')^{-1}}.
\end{equation*}
Take $\Group = \QR$, $p = 2q + 1$, with $p$ and $q$ primes.
Then $\exists (b - b')^{-1}$.
\begin{align*}
	\H_{\pk}(x_1, x_2) & = y^{x_1} g^{x_2} \\
	\H_{\pk} & : \Integers_q^2 \to \Group.
\end{align*}
